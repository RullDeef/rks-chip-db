\chapter{Основная часть}
\section{МОБД по ЭКБ КП}
Модернизируемая отраслевая БД по ЭКБ КП состоит из следующих модулей(подсистем):
\begin{itemize}
	\item хранилища данных;
	\item подсистемы сбора и учета данных;
	\item подсистемы формирования выходных отчетов;
	\item подсистемы информационной безопасности;
	\item подсистемы администрирования.
\end{itemize}
Хранилище данных предназначено для накопления и хранения данных элементов ЭКБ. Хранилище реализовано в виде базы данных. Базой данных является представленная в объективной форме совокупность самостоятельных материалов, систематизированных таким образом, чтобы эти материалы могли быть найдены и обработаны с помощью электронной вычислительной машины (ЭВМ). \cite{bd}

Подсистема сбора и учета данных предназначена для сбора и загрузки в Модернизированную отраслевую БД по ЭКБ КП, а также выгрузки из нее информации, необходимой для автоматизации процесса оценки и выбора элементов ЭКБ КП. 

Подсистема информационной безопасности предназначена для более обширного спектра задач. Пользователь имеет возможность аутенитфткации с помощью программных средств. Также подсистема информационной безопасности обеспечивает защиту ПАК от несанкционированного доступа к БД по ЭКБ КП.

Подсистема администрирования обеспечивает контроль за действиями пользователей в системе, создание и удаление пользователей, возможность разграничения доступа  пользователей к данным.

\section{Архитектура хранилища МОБД по ЭКБ КП}
Хранилище данных МОБД по ЭКБ КП реализовано посредством технологии <<клиент-сервер>>.

\section{Детали реализации}
Подсистемы сбора и учета данных, формирования выходных отчетов, информационной безопасности и администрирования реализованы на языке программирования Java, используя платформу для разработки бизнес-приложений CUBA. \cite{cuba} 
Платформа имеет возможность нативной поддержки PostgreSQL, \cite{psql} которая используется в качестве системы управления базами данных для базы данных элементов ЭКБ.

\subsection{Версионирование и поддержка CUBA}

Нумерация стабильных версий CUBA Platform формируется в соответствии с традиционным семантическим версионированием \cite{cuba-ver}:
\begin{equation*}
	major.minor.maintenance,
\end{equation*}
где:
\begin{itemize}
	\item  $maintenance$ -- обновление устранения неполадок. Обеспечивает обратную совместимость. Включает незначительные дополнительные возможности или улучшения, исправление дефектов, незначительные обновления, критические обновления для производительности и безопасности. Такие обновления не несут существенных изменений.
	\item $minor$ --  обновление, в основном совместимое с предыдущими версиями, однако может привносить существенные изменения на уровне основных возможностей. Предназначение $minor$-релиза -- введение новых возможностей при быстром процессе обновления.
	\item $major$ -- основное обновление. Включает в себя несовместимые изменения базовой архитектуры, функциональных возможностей, изменения на уровне программного интерфейса приложения, лежащего в основе библиотек и их версий. Для основных обновлений обратная совместимость необязательна.
\end{itemize}
ПАК МОБД по ЭКБ КП -- проект с длительным циклом обновления. Поэтому подсистемы реализованы с использованием версии, для которой осуществляется только корпоративная поддержка. Для того, чтобы компоненты приложения могли функционировать в системе, необходим приватный репозиторий артефактов.

\subsection{Sonatype Nexus}
Sonatype Nexus -- интегрированная платформа, с помощью которой разработчики могут хранить и управлять локальными зависимостями Java (Maven). Выбор платформы обусловлен тем, что локальные артефакты, хранимые в репозитории недоступны из внешних репозиториев.

\section{Набор клиентских средств для выкладки ПАК}

Конфигурация среды разработки зависит от платформы, на которой происходит эта самая разработка. Поэтому решено использовать механизм контейнеризации, чтобы абстрагироваться от имеющегося окружения и работать с неизменной и защищенной средой внутри контейнера.

Docker \cite{docker} -- это платформа контейнеризации с открытым исходным кодом, с помощью которой можно автоматизировать создание приложений, их доставку и управление.

\subsection{Компоненты Docker, используемые в выкладке} 
В таблице \ref{tab:docker-ref} даны определения основных компонентов Docker, использованных при выкладке.
\begin{table}[H]
	\centering
	\captionsetup{singlelinecheck = false, justification=raggedleft}
	\caption{Компоненты Docker, используемые в выкладке}
	\begin{tabular}{|l|l|l|}
		\hline
		Имя компонента     & Описание                                                                                                                                           & Применение                                                                                                             \\ \hline
		Dockerfile         & \begin{tabular}[c]{@{}l@{}}Текстовый файл с\\ последовательно \\ расположенными\\ инструкциями для \\ создания образа Docker\end{tabular}          & \begin{tabular}[c]{@{}l@{}}Развертывание \\ окружения, \\ необходимого\\ для сборки артефактов\end{tabular}            \\ \hline
		image              & \begin{tabular}[c]{@{}l@{}}Неизменяемый файл (образ),\\  из которого можно\\ неограниченное количество\\  раз развернуть контейнер\end{tabular}    & \begin{tabular}[c]{@{}l@{}}Использованы образы \\ окружения сборки,\\ репозитория Nexus\end{tabular}                   \\ \hline
		docker-compose.yml & \begin{tabular}[c]{@{}l@{}}Определение служб для \\ централизованного запуска\\ при сборке \\ многоконтейнерного \\ приложения Docker\end{tabular} & \begin{tabular}[c]{@{}l@{}}Централизованный \\ запуск служб Nexus, \\ СУБД Postgres и \\ окружения сборки\end{tabular} \\ \hline
		docker volumes     & \begin{tabular}[c]{@{}l@{}}Тома для постоянного\\  хранения информации\end{tabular}                                                                & \begin{tabular}[c]{@{}l@{}}Хранение \\ изменяемых данных\\ после остановки \\ службы\end{tabular}                      \\ \hline
	\end{tabular}
	\label{tab:docker-ref}
\end{table}

\section{Процесс выкладки программного окружения}
На листинге \ref{lst:compose} приведено определение служб в файле docker-compose для централизованного запуска при сборке многоконтейнерного приложения Docker.
\captionsetup{singlelinecheck = false, justification=raggedright}
\begin{lstlisting}[label=lst:compose,caption=файл docker-compose.yml]
version: '3.7'

services:
  nexus:
    build: nexus
    container_name: nexus
    ports:
      - '8081:8081'
    volumes:
      - ./nexus/data:/nexus-data
  database:
    image: postgres:10-alpine
    container_name: database
    ports:
      - '5432:5432'
    volumes:
      - ./database/data:/var/lib/postgresql/data
    environment:
      - POSTGRES_PASSWORD=postgres
  builder:
    build: builder
    container_name: builder
    network_mode: host
    volumes:
      - ./sources:/home/gradle
    depends_on:
      - nexus
      - database
\end{lstlisting}

Корневой ключ в этом файле -- services. Под этим ключом определяются службы, которые требуется развернуть и запустить. В данном случае в файле docker-compose.yml определено несколько служб, как показано в таблице \ref{tab:serv}.
\begin{table}[H]
	\centering
	\captionsetup{singlelinecheck = false, justification=raggedleft}
	\caption{Назначение служб}
	\begin{tabular}{|l|l|}
		\hline
		Имя службы & Описание                                                                                                                                                                                   \\ \hline
		nexus      & \begin{tabular}[c]{@{}l@{}}Организация локального хранилища пакетов\\ и артефактов CUBA, а также управление\\ запросами и перенаправление их на \\ специальные датацентры РКС\end{tabular} \\ \hline
		database   & \begin{tabular}[c]{@{}l@{}}Развертывание реляционной СУБД\\ PostgreSQL\end{tabular}                                                                                                        \\ \hline
		builder    & \begin{tabular}[c]{@{}l@{}}Окружение, необходимое для корректной\\ сборки артефактов системы\end{tabular}                                                                                  \\ \hline
	\end{tabular}
	\label{tab:serv}
\end{table}

\subsection[Окружение, необходимое для сборки артефактов]{Окружение, необходимое для сборки\\артефактов}

На листинге \ref{lst:builder} приведено содержание файла Dockerfile для развертывания окружения, необходимого для сборки артефактов.
\begin{lstlisting}[label=lst:builder,caption=Dockerfile для для службы builder]
FROM eclipse-temurin:8-jdk-jammy
	
CMD ["bash"]
	
ENV GRADLE_HOME /opt/gradle
	
RUN set -o errexit -o nounset \
	&& echo "Adding gradle user and group" \
	&& groupadd --system --gid 1000 gradle \
	&& useradd --system --gid gradle --uid 1000 ... \
	&& mkdir /home/gradle/.gradle \
	&& chown --recursive gradle:gradle /home/gradle \
	\
	&& echo "Symlinking root Gradle cache to gradle Gradle cache" \
	&& ln --symbolic /home/gradle/.gradle /root/.gradle
	
VOLUME /home/gradle/.gradle
	
WORKDIR /home/gradle
	
RUN set -o errexit -o nounset \
	&& apt-get update \
	&& apt-get install --yes --no-install-recommends \
		unzip \
		wget \
		\
		bzr \
		git \
		git-lfs \
		mercurial \
		openssh-client \
		subversion \
	&& rm --recursive --force /var/lib/apt/lists/* \
	\
	&& echo "Testing VCSes" \
	&& which bzr \
	&& which git \
	&& which git-lfs \
	&& which hg \
	&& which svn
	
ENV GRADLE_VERSION 4.3.1
ARG GRADLE_DOWNLOAD_SHA256=15ebe098ce0392a2d06...
RUN set -o errexit -o nounset \
	&& echo "Downloading Gradle" \
	&& wget --no-verbose --output-document=gradle.zip ... \
	\
	&& echo "Checking download hash" \
	&& echo "${GRADLE_DOWNLOAD_SHA256} *gradle.zip" | sha256sum --check - \
	\
	&& echo "Installing Gradle" \
	&& unzip gradle.zip \
	&& rm gradle.zip \
	&& mv "gradle-${GRADLE_VERSION}" "${GRADLE_HOME}/" \
	&& ln --symbolic "${GRADLE_HOME}/bin/gradle" /usr/bin/gradle \
	\
	&& echo "Testing Gradle installation" \
	&& gradle --version
\end{lstlisting}

\subsection{Сборка подсистем отраслевой БД по ЭКБ КП}
После успешного запуска всех необходимых служб с помощью задач в системе автоматической сборки <<Gradle>> осуществляется сборка подсистем в строгом порядке, поскольку на каждом этапе определенная подсистема отправляет в удаленный репозиторий Nexus артефакты, требуемые для дальнейшей выкладки. 

Основной модуль запускается с помощью <<Apache Tomcat>> -- программы, представляющая собой сервер, который занимается системной поддержкой сервлетов и обеспечивает их жизненный цикл в соответствии с правилами, определёнными в спецификациях. \cite{tomcat} 

Выкладка программного комплекса на данном этапе завершена и программное обеспечение ПАК МОБД ЭКБ доступно по адресу, указанному в файле docker-compose для централизованного запуска служб приложения (листинг \ref{lst:compose}).