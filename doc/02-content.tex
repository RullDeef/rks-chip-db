\chapter{Основная часть}
\section{МОБД по ЭКБ КП}
Модернизируемая отраслевая БД по ЭКБ КП состоит из следующих модулей(подсистем):
\begin{itemize}
	\item хранилища данных;
	\item подсистемы сбора и учета данных;
	\item подсистемы формирования выходных отчетов;
	\item подсистемы информационной безопасности;
	\item подсистемы администрирования.
\end{itemize}
Хранилище данных предназначено для накопления и хранения данных элементов ЭКБ. Хранилище реализовано в виде базы данных. Базой данных является представленная в объективной форме совокупность самостоятельных материалов, систематизированных таким образом, чтобы эти материалы могли быть найдены и обработаны с помощью электронной вычислительной машины (ЭВМ). \cite{bd}

Подсистема сбора и учета данных предназначена для сбора и загрузки в Модернизированную отраслевую БД по ЭКБ КП, а также выгрузки из нее информации, необходимой для автоматизации процесса оценки и выбора элементов ЭКБ КП. 

Подсистема информационной безопасности предназначена для более обширного спектра задач. Пользователь имеет возможность аутенитфткации с помощью программных средств. Также подсистема информационной безопасности обеспечивает защиту ПАК от несанкционированного доступа к БД по ЭКБ КП.

Подсистема администрирования обеспечивает контроль за действиями пользователей в системе, создание и удаление пользователей, возможность разграничения доступа  пользователей к данным.

\section{Архитектура хранилища МОБД по ЭКБ КП}
Хранилище данных МОБД по ЭКБ КП реализовано посредством технологии <<клиент-сервер>>.

\section{Детали реализации}
Подсистемы сбора и учета данных, формирования выходных отчетов, информационной безопасности и администрирования реализованы на языке программирования Java, используя платформу для разработки бизнес-приложений CUBA. \cite{cuba} 
Платформа имеет возможность нативной поддержки PostgreSQL, \cite{psql} которая используется в качестве системы управления базами данных для базы данных элементов ЭКБ.

\subsection{Версионирование и поддержка CUBA}

Нумерация стабильных версий CUBA Platform формируется в соответствии с традиционным семантическим версионированием \cite{cuba-ver}:
\begin{equation*}
	major.minor.maintenance,
\end{equation*}
где:
\begin{itemize}
	\item  $maintenance$ -- обновление устранения неполадок. Обеспечивает обратную совместимость. Включает незначительные дополнительные возможности или улучшения, исправление дефектов, незначительные обновления, критические обновления для производительности и безопасности. Такие обновления не несут существенных изменений.
	\item $minor$ --  обновление, в основном совместимое с предыдущими версиями, однако может привносить существенные изменения на уровне основных возможностей. Предназначение $minor$-релиза -- введение новых возможностей при быстром процессе обновления.
	\item $major$ -- основное обновление. Включает в себя несовместимые изменения базовой архитектуры, функциональных возможностей, изменения на уровне программного интерфейса приложения, лежащего в основе библиотек и их версий. Для основных обновлений обратная совместимость необязательна.
\end{itemize}
ПАК МОБД по ЭКБ КП -- проект с длительным циклом обновления. Поэтому подсистемы реализованы с использованием версии, для которой осуществляется только корпоративная поддержка. Для того, чтобы компоненты приложения могли функционировать в системе, необходим приватный репозиторий артефактов.

\subsection{Sonatype Nexus}
Sonatype Nexus -- интегрированная платформа, с помощью которой разработчики могут хранить и управлять локальными зависимостями Java (Maven). Выбор платформы обусловлен тем, что локальные артефакты, хранимые в репозитории недоступны из внешних репозиториев.

\section{Набор клиентских средств для выкладки ПАК}

пердокер

\section{Процесс выкладки программного окружения}

лешины драгоценные скрипты


